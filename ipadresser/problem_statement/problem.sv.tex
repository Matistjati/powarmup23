\problemname{IP-adresser}
En IPv4-address består av fyra heltal mellan $0$ och $255$ (som vardera inte får ha några inledande nollor), separerade av punkter.
T.ex. är \texttt{1.0.3.255} en giltig address, medan \texttt{1.0.03.255}, \texttt{1.0.3.256} och \texttt{1.0.3} inte är giltiga addresser.
Givet en sekvens av siffror, hitta alla giltiga IPv4-adresser som kan skapas genom insättning av tre punkter i sekvensen.

\section*{Indata}
På första raden står en sträng med minst $4$ och max $12$ siffror.

\section*{Utdata}
Skriv ut ett heltal: antalet giltiga IP-adresser som kan bildas genom att sätta in 3 punkter.


\noindent
\begin{tabular}{| l | l | p{12cm} |}
  \hline
  Grupp & Poängvärde & Gränser \\ \hline
  $1$    & $30$        & Indata innehåller inga nollor. \\ \hline 
  $2$    & $70$        & Inga ytterligare begränsingar. \\ \hline 
\end{tabular}


\section*{Förklaring av exempelfall}
Förklaring av exempelfall $1$: Det finns endast en giltig ip-address som kan bildas, \texttt{255.255.255.255}. Alla andra utsättningar av punkter ger tal som är längre än $3$.

Förklaring av exempelfall $2$: Eftersom det finns $4$ siffror finns bara en giltig utsättning punkter. Eftersom siffran noll är tillåten blir svaret \texttt{0.0.0.0}.

Förklaring av exempelfall $3$: Hur punkterna än placers ut kommer det alltid finnas tal med inledande nollor som inte är siffran noll. Därför blir svaret $0$.

Förklaring av exempelfall $4$: $7$ strängar kan bildas$\colon$
\begin{enumerate}
    \item \texttt{2.9.18.41}
    \item \texttt{2.9.184.1}
    \item \texttt{2.91.8.41}
    \item \texttt{2.91.84.1}
    \item \texttt{29.1.8.41}
    \item \texttt{29.1.84.1}
    \item \texttt{29.18.4.1}
\end{enumerate}
