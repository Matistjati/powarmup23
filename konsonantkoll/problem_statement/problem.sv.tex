\problemname{Konsonantkoll}
Din kompis Frida älskar konsonanter! Tyvärr ställer det här till problem för henne när hon skriver på datorn. Hon gillar nämligen konsonanter så mycket att Frida, när det ska stå två av samma konsonant på raken, i sin iver ibland råkar trycka på tangenten allldeles för många gånger.

För att hjälpa Frida ska du skriva ett program som tar bort de extra konsonanterna. I svenska finns 20 konsonanter: \textbf{bcdfghjklmnpqrstvwxz}. Praktiskt nog så förekommer aldrig tre eller fler av samma konsonant på raken i korrekt text.

\section*{Indata}
Indata består av exakt en rad, som bara kommer innehålla små bokstäver (a-z) och eventuellt mellanslag. Mellanslag kommer aldrig finnas i början eller slutet av raden, och det kommer aldrig vara flera på raken (precis som i en vanlig mening). Raden är mellan 1 och 1000 tecken lång.

\section*{Utdata}
Skriv ut samma rad med överflödiga konsonanter borttagna, så att det aldrig finns fler än två av samma konsonant i följd.

\section*{Poängsättning}
Din lösning kommer att testas på en mängd testfallsgrupper.
För att få poäng för en grupp så måste du klara alla testfall i gruppen.

\noindent
\begin{tabular}{| l | l | l |}
\hline
Grupp & Poängvärde & Begränsningar \\ \hline
1     & 50          &  Det förekommer inga mellanslag, och inte fler än 3 av samma konsonant i följd.\\ \hline
2     & 50         &  Inga begränsningar. \\ \hline
\end{tabular}

